% Source: http://tex.stackexchange.com/a/5374/23931
\documentclass{article}
\usepackage[T1]{fontenc}
\usepackage[utf8]{inputenc}
\usepackage[margin=1in]{geometry}
\usepackage{hyperref}
\usepackage{biblatex}
\addbibresource{newbbib.bib}

\newcommand{\HRule}{\rule{\linewidth}{0.5mm}}
\newcommand{\Hrule}{\rule{\linewidth}{0.3mm}}

\makeatletter% since there's an at-sign (@) in the command name
\renewcommand{\@maketitle}{%
  \parindent=0pt% don't indent paragraphs in the title block
  \centering
  {\Large \bfseries\textsc{\@title}}
  \HRule\par%
  \textit{\@author \hfill \@date}
  \par
}
\makeatother% resets the meaning of the at-sign (@)

\title{Statement of Purpose}
\author{Spencer Edwards}
\date{9/7/2020}

\begin{document}
  \maketitle% prints the title block

\hypertarget{introduction}{%
\subsection{Introduction}\label{introduction}}

The past \emph{three} years at Western Washington University have had a
massive impact on my career and research goals. When I came to WWU I
knew I wanted to study ``the oceans'', but I lacked the knowledge and
experience to focus my interest to a manageable field of study. As is
the case with most general interests, the more I studied the more I
realized just how much I didn't know. However, each year at Western, and
each class I took, helped me narrow down my interests and create a more
focused, manageable career plan. I have had the opportunity to
participate in lab research, and classes that I took in genetics and
ecology have cemented my desire to continue my career in the research
field. Seminars and independent research have honed my interests in
phenotypic plasticity and ecology as it relates to climate change. The
result of my experiences and formal education at WWU is a keen interest
in using methods in evolutionary biology and genetics to understand
organismal adaptations and applying that understanding to ecological
problems.

\hypertarget{experience}{%
\subsection{Experience}\label{experience}}

\hypertarget{freshman-year}{%
\paragraph{Freshman Year}\label{freshman-year}}

In my freshman year, I assisted Geoffrey Mayhew with his thesis in the
Miner lab, examining phenotypic plasticity in the hatch timing of
Eastern Pacific Nudibranchs. Work was performed both in-lab and in the
field, where specimens were collected. While working in the lab I
learned techniques for distilling and filtering seawater, collecting egg
masses quickly and delicately, as well as obtaining specimens in the
field. While this research did involve looking at plasticity, it wasn't
until later that I realized I could connect this concept to my own
interests.

\hypertarget{sophomore-year}{%
\paragraph{Sophomore Year}\label{sophomore-year}}

In my Sophomore year of college, I became interested in performing
research on climate change. Of course, I had known climate change was an
important issue,however until then I had not considered explicitly
studying biology with an emphasis on climatic change. It was also in
this year that, while watching seminars given by researchers presented
by Western's biology department, that I became interested in ecology and
genetics. Specifically, a talk by Dr.~Emily Grayson on the genetics and
ecology of the invasive green crab,. Her work with Washington Sea Grant
showed that the crab had formed genetically distinct metapopulations in
Washington State. This made me interested in the ecology of invasive
species. It was during this year I became interested in applying ecology
to climate change, analyzing how populations and communities would
respond. During the summer, I took genetics, and discovered an interest
in genomics, and how we use genetics to understand ecology.

\hypertarget{junior-year}{%
\paragraph{Junior Year}\label{junior-year}}

In my Junior year, I was fortunate to take classes from across the
different fields of biology, as well as more marine-oriented classes.
These included ecology, evolutionary biology,as well as a class on algal
ecology. It was these 3 classes that had the largest impact on my
thinking. Over the course of the year, I began to seriously consider
autecology (the ecology of individual organisms), and how evolution and
genetics could help us answer questions about the connection between
individuals and the ``big picture'' of communities and entire
ecosystems. It was in the spring of my Junior year, when I took the
algal ecology class, that I rediscovered the concept of plasticity.
Whereas in my Freshman year,I had not really understood the full
importance of what I was working on, my Junior year the importance hit
me right in the face. The more I read about the ecology and development
of algae, in particular, the invasive species \emph{Sargassum muticum},
the more I realized that plasticity could be a major mechanism that
allowed organisms to adapt to novel environments. Indeed, far from being
the bizarre novelty I had thought it was, developmental plasticity was
exhibited by most organisms in which it had been studied, and appeared
to be a major contributing factor to rapid acclimation and evolution.
After the end of the school year, I continued to do research, and that
was when I discovered the concept of eco-evo-devo.

\hypertarget{summer-of-junior-year}{%
\paragraph{Summer of Junior Year}\label{summer-of-junior-year}}

Over this past summer, while researching plasticity in arctic organisms,
I came across a paper entitled \emph{Eco-Evo-Devo: the Time has Come}.
The paper, published in 2014, described the emerging methods and modern
synthesis of these three fields.

\begin{quote}
``Combining the approaches of eco-evo-devo and ecological genomics will
mutually enrich these fields in a way that will not only enhance our
understanding of evolution, but also of the genetic mechanisms
underlying the responses of organisms to their natural environments''
\autocite{Abouheif2014}.
\end{quote}

This was the light bulb moment. I realized that not only was there an
entire field that was incorporating the concepts that I had wanted to
combine, but also, the field was extremely new, and thus presented many
new questions that have yet to be answered. This new field, formally
known as ecological evolutionary developmental biology (abbreviated
eco-evo-devo) allows us to apply principles from late 20th and early
21st century evolutionary developmental biology to current ecology. For
instance, by analyzing the role that genetics and developmental
plasticity combined play in evolution, we can predict the response of
organisms/communities to global environmental change. In particular, I
want to understand how polar organisms can potentially adapt to cope
with the effects of climate change, as well as how we can predict which
species will thrive in Arctic and Antarctic climates, and which could
potentially lose out. The poles are of particular interest as not only
the epicenter for global climate change, but as they have emerged
relatively recently, adaptations in these ecosystems are newer and
potentially more resilient to change. Conversely, the relative lack of
diversity could be a detriment to any adaptations that might arise due
to poor gene flow. Understanding these contrasting forces, as well as
the adaptations of organisms that have risen in the past (and
currently), and predicting their response to climate change will allow
for intelligent methods for conservation and policy recommendations.

\hypertarget{concluding-remarks}{%
\subsection{Concluding Remarks}\label{concluding-remarks}}

Western Washington University has provided me with not only the
knowledge base to understand modern biological problems, but also the
experiences that have led me to synthesize my interests. Research
experience taught me that I was very passionate about lab work, and the
classes I took gave me background knowledge I needed to ask questions
that could be researched. Ecology, Genetics, Evolutionary Biology, and
Algal Ecology have all crafted my interest in broad fields within
biology, and my own personal research has helped me to discover a novel
field in which to combine my interests and carry out research.


\printbibliography



\end{document}