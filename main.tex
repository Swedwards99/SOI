%!TEX program = xelatex
%%%%%%%%%%%%%%%%%%%%%%%%%%%%%%%%%%%%%%%%%%%%%%%%%%%%%%%%%%%%%%%%%%%%%
%% Title: SOP LaTeX Template
%% Author: Soonho Kong / soonhok@cs.cmu.edu
%% Created: 2012-11-12
%%%%%%%%%%%%%%%%%%%%%%%%%%%%%%%%%%%%%%%%%%%%%%%%%%%%%%%%%%%%%%%%%%%%%

%%%%%%%%%%%%%%%%%%%%%%%%%%%%%%%%%%%%%%%%%%%%%%%%%%%%%%%%%%%%%%%%%%%%%
%%
%% Requirement:
%%     You need to have the `Adobe Caslon Pro` font family.
%%     For more information, please visit:
%%     http://store1.adobe.com/cfusion/store/html/index.cfm?store=OLS-US&event=displayFontPackage&code=1712
%%
%% How to Compile:
 %$ xelatex main.tex
%%
%%%%%%%%%%%%%%%%%%%%%%%%%%%%%%%%%%%%%%%%%%%%%%%%%%%%%%%%%%%%%%%%%%%%%

\documentclass[letterpaper]{article}
\usepackage[letterpaper,margin=1.75in,noheadfoot]{geometry}
\usepackage{fontspec, color, enumerate, sectsty}
\usepackage[normalem]{ulem}

%%%%%%%%%%%%%%%%%%%%%%%%%%%%%%%%%%%%%%%%%%%%%%%%%%%%%%%%%%%%%%%%%%%%%
%                      YOUR INFORMATION
%
%      PLEASE EDIT THE FOLLOWING LINES ACCORDINGLY!!
%%%%%%%%%%%%%%%%%%%%%%%%%%%%%%%%%%%%%%%%%%%%%%%%%%%%%%%%%%%%%%%%%%%%%
\newcommand{\soptitle}{Statement of Purpose}
\newcommand{\yourname}{Spencer Edwards}
\newcommand{\youremail}{edward63@wwu.edu}

%% FONTS SETUP
\defaultfontfeatures{Mapping=tex-text}
\setromanfont[Path = fonts/, Ligatures={Common}]{adobe_caslon_pro}
\setmonofont[Path = fonts/, Scale=0.8]{monaco}
\setsansfont[Path = fonts/, Scale=0.9]{Optima-Regular}
\newcommand{\amper}{{\fontspec[Scale=.95]{Adobe Caslon Pro}\selectfont\itshape\&~{}}}
\usepackage[bookmarks, colorlinks, breaklinks,
pdftitle={\yourname - \soptitle},pdfauthor={\yourname}, unicode]{hyperref}
\hypersetup{linkcolor=magneta,citecolor=magenta,filecolor=magenta,urlcolor=[named]{WildStrawberry}}

%%%%%%%%%%%%%%%%%%%%%%%%%%%%%%%%%%%%%%%%%%%%%%%%%%%%%%%%%%%%%%%%%%%%%
%                      Title and Author Name
%%%%%%%%%%%%%%%%%%%%%%%%%%%%%%%%%%%%%%%%%%%%%%%%%%%%%%%%%%%%%%%%%%%%%
\begin{document}
\begin{center}{\huge \scshape \soptitle}\end{center}
\begin{center}\vspace{0.2em} {\Large \yourname\\}
  {\youremail}\end{center}

%%%%%%%%%%%%%%%%%%%%%%%%%%%%%%%%%%%%%%%%%%%%%%%%%%%%%%%%%%%%%%%%%%%%%
%                      SOP Body
% NOTE: Use \amper instead of \&
%%%%%%%%%%%%%%%%%%%%%%%%%%%%%%%%%%%%%%%%%%%%%%%%%%%%%%%%%%%%%%%%%%%%%
\section*{Introduction}
The past four years at Western Washington University have had a massive impact 
on my career and research goals. When I came into WWU I knew I wanted to study 
"the oceans", but not much else. Each year at Western, and each class I took 
helped me narrow down my interests and create a solid career plan. 
Specifically, I'm interested in using methods in evolutionary biology and genetics 
to understand organismal adaptations and applying that understanding to ecological problems. The new field of ecological evolutionary developmental biology
(eco-evo-devo) allows us to apply principles from late 20th and early 21st century evolutionary developmental biology to current ecology. For instance, by analyzing the role
that genetics and developmental plasticity combined play in evolution, we can predict the response of organisms/communities 
to global environmental change.
In particular, I want to understand how polar organisms can potentially adapt to cope with the effects of climate change,
as well as how we can predict which species will thrive in arctic and antarctic climates, and which will lose out. 
The poles are of particular interest as not only the 
epicenter for global climate change, but as they have emerged relatively recently, 
adaptations to these ecosystems are newer and potentially more resilient to change. Conversely, the relative lack of diversity could be a detriment to
any adaptations that might arise due to poor gene flow.  
Understanding these contrasting forces, as well as the adaptations of organisms that have risen in the past (and currently), and predicting their response to climate 
change will allow for intelligent methods for conservation and policy recommendations. 
\section*{Experience}
\paragraph{Freshman Year}
In my freshman year, I assisted Geoffrey Mayhew with his thesis in the Miner lab, 
examining phenotypic plasticity in the hatch timing of Eastern Pacific Nudibranchs. Work was
 performed both in-lab and in the field, where specimens were collected.  While working in the lab 
I learned techniques for distilling and filtering seawater, collecting egg masses 
quickly and delicately, as well as obtaining specimens in the field.
While this research did involve looking at plasticity, it wasn't until later that I realized I could connect 
this concept to my own interests. 

\newpage

\paragraph{Sophmore Year}
In my Sophomore year of college, I became interested in performing research on climate change. Of course, I had known climate change was an important issue,
however until then I had not considered explicitly studying biology with an emphasis on climatic change. It was also in this year that, while watching seminars given by researchers
presented by Western's biology department, that I became interested in ecology and genetics. Specifically, a talk by Dr. Emily Grayson
on the genetics and ecology of the invasive green crab,. Her work with Washington Sea Grant showed that the crab
had formed genetically distinct metapopulations in Washington State. This made me interested in the ecology of invasive species.
It was during this year I became interested in applying ecology to climate change, analyzing how populations and communities would respond.

\paragraph{Junior Year}
In my Junior year, I was fortunate to take classes from across the different fields of biology, as well as more marine-oriented classes. These included ecology, evolutionary biology,
as well as a class on algal ecology. It was these 3 classes that had the largest impact on my thinking. 
Over the course of the year, I began to seriously consider autecology (the ecology of individual organisms), and how evolution and genetics could help us answer questions about the connection between individuals and the "big picture" of communities and 
entire ecosystems. It was in the spring of my Junior year, when I took the algal ecology class, that I rediscovered the concept of plasticity. Whereas in my Freshman year,
I had not really understood the full importance of what I was working on,  my Junior year the importance hit me right in the face. The more I read about the ecology and development of algae, in particular, the invasive species
\textit{Sargassum muticum}, the more I realized that plasticity could be a major mechanism that allowed organisms to adapt to novel environments. Indeed, far from being the bizarre novelty I had thought it was, developmental plasticity was exhibited by most organisms in which it had been studied,
and appeared to be a major contributing factor to rapid acclimation and evolution. After the end of the school year, I continued to do research, and that was when I discovered the concept of eco-evo-devo.
\paragraph{Summer of Junior Year}
Over this past summer, while researching plasticity in arctic organisms, I came across a paper entitled \textit{Eco-Evo-Devo: the Time has Come}. The paper, published in 2014, described the emerging methods and modern synthesis of these three fields.
 "Combining the approaches of eco-evo-devo and ecological genomics will mutually enrich these fields in a way that will not only enhance our understanding of evolution, but also of the genetic mechanisms underlying the responses of organisms to their 
 natural environments". 
This was the lightbulb moment. I realized that not only was there an entire field that was incorporating the concepts that I had wanted to combine, but also, the field was extremley new, and thus presented many new questions that
have yet to be answered. Applying eco-evo-devo to climate change, along with performing long term ecological research, could potentially allow us to not only understand how organisms will adapt, but could provide insights into crop management and ecosystem restoration.

\section*{Concluding Remarks}
Western Washington University has provided me with not only the knowledge base to understand modern biological problems, but also the experiences that have led me to synthesize my interests.  
\end{document}